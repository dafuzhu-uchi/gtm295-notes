\setcounter{section}{8}
\section{Independence}

\begin{note}{p.~170}
    Given $X_j$ is in $(E_j,\mathcal{E}_j)$. The following expressions are equivalent:
    \begin{enumerate}
        \item $\mathbb{P}(A_j)$: $A_j$ is an element in $\sigma(X_j)$
        \item $\mathbb{P}(\{X_j\in F_j\})$: $F_j$ is an element in $\mathcal{E}_j$
    \end{enumerate}
    In (1), $\sigma(X_j)=\{X_j^{-1}(F):F\in\mathcal{E}_j\}$. An element in $\sigma(X_j)$ means for some $F_j\in\mathcal{E}_j$, $A_i=\{X_j^{-1}(F_j)\}=\{\omega\in\Omega:X_j(\omega)\in F_j\}$. This is exactly $\{X_j\in F_j\}$ in (2).
\end{note}

\begin{note}{Thm 9.4}
    In Thm 5.2(1), $\mu\otimes v(A\times B)=\mu(A)v(B)$, so 
    \[
    \mathbb{P}_{X_{1}} \otimes \cdots \otimes \mathbb{P}_{X_{n}}\left(F_{1} \times \cdots \times F_{n}\right)=\prod_{i=1}^{n} \mathbb{P}_{X_{i}}\left(F_{i}\right)
    \]
\end{note}